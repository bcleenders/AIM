\section{Introduction}
\subsection{What is Hacker News}
Hacker News is a social news site: it aggregates news by allowing users to submit stories. Interesting submissions can be upvoted by other users and all submissions are ranked by popularity.

The content on Hacker News is mostly related to science, in particular computer science. The guidelines for what content can be posted are very broad; the guidelines specify on topic as ``\textit{anything that gratifies one's intellectual curiosity}''\footnote{\url{https://news.ycombinator.com/newsguidelines.html}}.

Hacker News started in February 2007, eight years ago at the time of writing, and has experienced rapid growth, resulting in a daily 2.6 million pageviews and 3.5 unique visitors per month\footnote{\url{https://news.ycombinator.com/item?id=9219581}}. One of the reasons suggested for this popularity is Hacker News' similarity to how Reddit used to be\footnote{\url{http://techcrunch.com/2013/05/18/the-evolution-of-hacker-news/}}: user-submitted content with a very minimalistic, terminal-like interface.

\subsection{Research Question}
A lot happened and changed during the last eight years, especially in the field of computer science. This research is ment to demonstrate how these real-life events are correlate with the activity on Hacker News.

This lead to the following research question:\\
\\
\textsc{Research Question:} what correlation can we find between real-world events and the popularity of corresponding topics on Hacker News?\\
% Answer: three types of events: single-time, recurring events and general trends
% More concrete -> example charts
\\
This question depends on two other questions, since we have not yet specified what we mean by popular nor what topics we mean exactly. These subquestions are:\\
\\
\textsc{SubQuestion 1:} what topics does the Hacker News content consist of?\\
\textsc{SubQuestion 2:} how does one quantify the popularity of a topic?

\subsection{Outline}
We start this paper with a description of the dataset and our data processing steps in section~\ref{sec:dataset}. In section~\ref{sec:exploratory} we give the results of some exploratory research with the dataset to give a feeling what data is like.

After the exploratory research, we describe how we clustered the articles. We used two methods to divide the articles into categories: Latent Dirichlet allocation and a combination of Word2Vec and k-means. Both methods and their implementations are explained in section~\ref{sec:topic_detection}. 

The results of the classification and the trends are shown in section~\ref{sec:results}.