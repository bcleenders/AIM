\section{Conclusion}
The goal of this research was to find out how the popularity of topics on Hacker News evolves over time. To answer this question, we defined a popularity measure based on the percentage of articles, upvotes and comments in a topic.

We then continued by dividing the dataset 1.5 million articles into topics using two unsupervised clustering algorithms: Word2Vec in combination with k-means and Latent Dirichlet allocation.

The resulting distribution of stories over these topics allowed us to determine the popularity of topics. Looking at the popularity plotted over time, we have seen three types of ``trends":
\begin{itemize}
\item Single events, e.g. deaths and product releases (Steve Jobs, Hyperloop). These cause big spikes during the month of the event, but don't have long-lasting effects.
\item Recurring evens, e.g. annual conferences (WWDC). These clearly show up in the results, although they are generally not as extreme compared to single-time events.
\item Long-lasting trends, e.g. concepts or technologies (Raspberry Pi, Docker/CoreOS). These topics attract attention over a longer span of time, but even within these time spans the months with big news (product releases) generally have a much higher (by a factor of two or three) popularity. Only really big topics have such a long-lasting trend.
\end{itemize}

