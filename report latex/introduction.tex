\section{Introduction}
\subsection{What is HackerNews}
Hacker News is a social news site: it aggregates news by allowing users to submit stories. Interesting submissions can be upvoted by other users and all submissions are ranked by popularity.

The content on Hacker News is mostly related to science, in particular computer science. The guidelines for what content can be posted are very broad; the guidelines specify on topic as ``\textit{anything that gratifies one's intellectual curiosity}''\footnote{\url{https://news.ycombinator.com/newsguidelines.html}}.

Over the last eight years Hacker News has experienced rapid growth, resulting in a daily 2.6 million pageviews and 3.5 unique visitors per month\footnote{\url{https://news.ycombinator.com/item?id=9219581}}. One of the reasons suggested for this popularity is Hacker News' similarity to how Reddis used to be\footnote{\url{http://techcrunch.com/2013/05/18/the-evolution-of-hacker-news/}}: user-submitted content with a very minimalistic, terminal-like interface.

\subsection{Research Question}
The world changed a lot over the last eight years, especially in the field of computer science.  The userbase of Hacker News and their interests have changed as well. In this research, we want to see how they changed. Informally phrased, we want to find trends in the popularity of several topics over the last eight years. This lead to the following research question:\\
\\
\textsc{Research Question:} how did the popularity of news topics on Hacker News change over time?\\
\\
This question depends on two other questions, since we have not yet specified what we mean by popular nor what topics we mean exactly. These subquestions are:\\
\\
\textsc{SubQuestion 1:} how does one quantify the popularity of a topic?\\
\textsc{SubQuestion 2:} what topics does the Hacker News content consist of?\\

\subsection{Outline}
We will address the first subquestion by ranking posts and users over the entire Hacker News history. This will help characterize the dataset and show how various ways of measurent often give similar results, and demonstrate some flaws in several ways of measuring popularity.

Dividing the content in topics is a more academically challenging task. We have used two methods to divide the articles into categories: latent Dirichlet allocation and a combination of Word2Vec and k-means. Both methods are explained in section~\ref{sec:topic_detection}. The results of this classification and the trends are shown in section~\ref{sec:results}.